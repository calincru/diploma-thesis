\chapter{Conclusion}

\paragraph{Summary.}

\paragraph{Future work.}
Because our efforts were targeted at having a feature-full implementation that
can take a real world iptables deployment and return a SEFL model, there is
still room for optimization in terms of the generated SEFL code.  One key
observation is that most of the verification time is spent in the underlying
satisfiability solver.  It is the main source of memory consumption too.
Therefore, fine-tuning our code to generated only the needed conditions on each
execution path (or as close to that as possible) can help dramatically speed up
symbolic execution.  However, this is by itself a very hard problem considering
the involved semantics of some iptables rules.

In addition to that, further integration testing needs to be performed.  In
this thesis, we limited our tests to synthetic rules which might miss certain
practices that are common in real deployments.  In fact, this is just a small
part of a larger project that aims to build a provably correct model of an
OpenStack deployment.  Therefore, integrating it with the other networking
components used as part of Neutron will be the immediate follow-up challenge.
